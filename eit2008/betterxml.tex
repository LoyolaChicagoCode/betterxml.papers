 %% bare_conf.tex
%% V1.3
%% 2007/01/11
%% by Michael Shell
%% See:
%% http://www.michaelshell.org/
%% for current contact information.
%%
%% This is a skeleton file demonstrating the use of IEEEtran.cls
%% (requires IEEEtran.cls version 1.7 or later) with an IEEE conference paper.
%%
%% Support sites:
%% http://www.michaelshell.org/tex/ieeetran/
%% http://www.ctan.org/tex-archive/macros/latex/contrib/IEEEtran/
%% and
%% http://www.ieee.org/

%%*************************************************************************
%% Legal Notice:
%% This code is offered as-is without any warranty either expressed or
%% implied; without even the implied warranty of MERCHANTABILITY or
%% FITNESS FOR A PARTICULAR PURPOSE! 
%% User assumes all risk.
%% In no event shall IEEE or any contributor to this code be liable for
%% any damages or losses, including, but not limited to, incidental,
%% consequential, or any other damages, resulting from the use or misuse
%% of any information contained here.
%%
%% All comments are the opinions of their respective authors and are not
%% necessarily endorsed by the IEEE.
%%
%% This work is distributed under the LaTeX Project Public License (LPPL)
%% ( http://www.latex-project.org/ ) version 1.3, and may be freely used,
%% distributed and modified. A copy of the LPPL, version 1.3, is included
%% in the base LaTeX documentation of all distributions of LaTeX released
%% 2003/12/01 or later.
%% Retain all contribution notices and credits.
%% ** Modified files should be clearly indicated as such, including  **
%% ** renaming them and changing author support contact information. **
%%
%% File list of work: IEEEtran.cls, IEEEtran_HOWTO.pdf, bare_adv.tex,
%%                    bare_conf.tex, bare_jrnl.tex, bare_jrnl_compsoc.tex
%%*************************************************************************

%
\documentclass[conference]{IEEEtran}
%\documentclass{IEEEconf}

  % \usepackage[pdftex]{graphicx}
  % declare the path(s) where your graphic files are
  % \graphicspath{{../pdf/}{../jpeg/}}
  % and their extensions so you won't have to specify these with
  % every instance of \includegraphics
  % \DeclareGraphicsExtensions{.pdf,.jpeg,.png}

\usepackage{listings}
%numbers=left,
%numbersep=0pt,
\lstset{language=Java,
        basicstyle=\ttfamily \footnotesize}

%\usepackage{algorithmic}

\begin{document}

\title{Taming XML: Objects First, Then Markup}

%\author{\IEEEauthorblockN{Matt Bone\IEEEauthorrefmark{1},
%Peter F. Nabicht\IEEEauthorrefmark{1},
%George K. Thiruvathukal\IEEEauthorrefmark{1}, and
%Konstantin Laufer\IEEEauthorrefmark{1}}
%\IEEEauthorblockA{\IEEEauthorrefmark{1}Emerging Technologies Laboratory\\
%Department of Computer Science\\
%Loyola University Chicago\\
%Chicago, IL 60640\\
%Email: \{mbone,nabicht,gkt,laufer\}@etl.luc.edu}
%}

\author{\IEEEauthorblockN{Matt Bone, Peter F. Nabicht, Konstantin L\"{a}ufer and George K. Thiruvathukal
\IEEEauthorblockA{Emerging Technologies Laboratory\\
Department of Computer Science\\
Loyola University Chicago\\
Chicago, IL 60640\\
\{mbone,nabicht,gkt,laufer\}@etl.luc.edu}\\
}
}

\hyphenation{XML NaturalXML}

\maketitle

\begin{abstract}
Processing markup in object-oriented languages often requires the
programmer to focus on the objects generating the markup rather than
the more pertinent domain objects.  The BetterXML framework aims to
improve this situation by allowing the programmer to develop a
domain-specific object model as usual and later bind this model to
preexisting or newly generated markup. To this end, the framework
provides two types of object trees, {\em XElement} and {\em
NaturalXML}, for representing XML documents.  {\em XElement} goes
beyond DOM-like automatic parsing of XML by supporting the custom
mapping of elements to domain objects; {\em NaturalXML} allows the
mapping of existing domain objects to XML elements using class
metadata.  Both types of object trees can be inflated and deflated by
means of a common intermediate representation in the form of an event
stream.  Finally, the framework includes the {\em XML Intermediate
Representation (XIR)}, a lossless record-oriented representation of
XML documents for efficient streaming and other types of data
exchange.

%\begin{itemize}
%\item {\em XElement} goes beyond DOM-like automatic parsing of XML
%  documents to an object tree by supporting the custom mapping of
%  elements to domain objects.
%\item {\em NaturalXML} allows the mapping of existing domain objects
%  to XML elements using Java annotations.
%\end{itemize} Both types of object trees can be inflated and deflated
%by means of a common intermediate representation in the form of event
%streams.

%XIR also serves
%as an external representation of event streams.

%The Java reference implementation of BetterXML, including the examples
%shown in this paper, is available as an open-source project through
%Google Code.
\end{abstract}

\section{Motivation and Approach}\label{sec:motivation}
When processing markup in an object-oriented language, we often
confront a large amount of accidental complexity.  Instead of dealing
with our domain objects directly, we are forced to consider event
streams or DOM trees.  The BetterXML framework aims to mitigate the
situation, allowing for the processing of markup without losing focus
on the underlying domain model. The design goals of BetterXML are ease
of use, simplicity, flexibility, and grounding in design patterns.

The BetterXML framework starts from the assumption that markup is
inherently tree-structured.  Though this starting point may differ
from the original intent of markup as a stream of characters
interrupted by the occasional tag, the XML specification does not
allow for documents without tags and requires exactly one root
element~\cite{xmlSpec}.  Thus the specification itself is a
recognition of this tree structure. Following this assumption,
BetterXML represents markup in the form of an object tree, and the
framework contributes two types of these trees, XElement and
NaturalXML, as well as an internal event stream representation and a
record-oriented external representation.

\paragraph*{XElement} While both types of object trees retain a
one-to-one relationship between elements and classes, XElement
is the more DOM-like of the two.  In XElement, all classes extend a
common base class.  This base class contains a single heterogeneous
list of all children and a map of attributes.

\paragraph*{NaturalXML} Classes in NaturalXML, on the other
hand, need not extend a common base class.  Instead the mapping
between elements, attributes, and child elements occurs in class
metadata, and each class is responsible for providing its own data
structures for children and attributes.  

\paragraph*{Event streams} Both tree formats in the framework know
nothing of the underlying markup they represent; instead, the trees
are inflated from and serialized to an event stream.  This stream,
though not meant for end users, is the lingua franca of the system,
serving as an intermediary that decouples the object trees from the
various kinds of markup the framework supports.

\paragraph*{XIR} Though the framework was originally intended for XML
processing, it is compatible with other forms of markup including its
own representation, the XML Intermediate Representation (XIR), a
record-oriented regular language that is easy to parse and for which
writing a parser is trivial. XIR also serves as an external
representation of event streams.

In this paper we will discuss BetterXML as it relates to the reference
Java implementation. The ideas of BetterXML, while closely tied to the
object-oriented paradigm, are independent of specific object-oriented
languages, and Python and C\#/.NET ports of the framework are in
progress.

\subsection{A Brief Example}
Throughout this paper we will use a simple applicative calculator to
illustrate various aspects of the BetterXML framework.  The calculator
will support integer addition and subtraction, and we will represent
these operations in XML.  For example, we express the computation
$21+20+(5-4)$ as:

\begin{lstlisting}
<group>
  <add>
    <value value="21"/>
    <value value="20"/>
    <subtract>
      <value value="5"/>
      <value value="4"/>
    </subtract>
  </add>
</group>
\end{lstlisting}
The computation, of course, evaluates to $42$.  If we were to represent
this as an expression tree in an object-oriented language, we would
most likely create a distinct class for each operation.  Each
of these classes might share a common method, \lstinline{evaluate()}.
Values would evaluate to themselves, and a more complex operation like
addition would call the \lstinline{evaluate()} method on all children
and return their sum.

This example is typical of an introductory object-oriented programming
or data structures course and is used to illuminate the ideas of
polymorphism and recursive evaluation.  In such a setting, the example
is valued for its conceptual clarity.  In the sections below we will
show how the BetterXML framework can represent this problem in XML
while using the aforementioned object model with little or no
modification.


\section{Object Trees}\label{sec:ootrees}
%%%%%%%%%%%%%%%%%%%%%%%%%%%%%%%%%%%%%%%%
% Object Trees                         %
%%%%%%%%%%%%%%%%%%%%%%%%%%%%%%%%%%%%%%%%

In this section, we discuss the two object tree formats provided by
BetterXML in more detail.

\subsection{XElement}
Of the two tree formats in BetterXML, XElement is the more DOM-like.
As in DOM, XElement classes must extend a common base class, and this
base class contains the data structures and methods for manipulating
children, attributes, and character data.  DOM, however, is not
designed for the programmer to create element-specific or other custom
subclasses. By contrast, XElement allows the programmer to map
specific XML elements to one particular subclass of the common base
class, thereby creating a domain-specific object tree.

For instance, in the calculator example we map the \lstinline'add',
\lstinline'subtract', \lstinline'group', and \lstinline'value'
elements to the \lstinline'Add', \lstinline'Subtract',
\lstinline'Group', and \lstinline'Value' classes respectively:
\begin{lstlisting}
ToXElementContentHandler handler = 
        new ToXElementContentHandler();  
handler.registerElementClass(Add.class, "add");
handler.registerElementClass(Subtract.class, 
                             "subtract");
handler.registerElementClass(Group.class, "group");
handler.registerElementClass(Value.class, value");  
\end{lstlisting}

Here, \lstinline'handler' is a reference to the event handler for the
BetterXML event stream (see section \ref{streams}), and this handler
is used to establish the mappings from element names to classes prior
to parsing a document. When the document is parsed and an element is
encountered, the mappings are examined.  If the element is mapped to a
particular class, then that class is instantiated.  Otherwise, the
base class, \lstinline'XElement', is instantiated.

Since all unmapped elements are represented as instances of 
\lstinline'XElement', the mappings from element names to classes
are optional.  Thus, smaller applications that lack a rich
domain model can rely directly on instances of \lstinline'XElement'.
The \lstinline'XElement' class itself is straightforward and has
methods for extracting the element name, attributes, and
children. Some methods of this class are excerpted below. (The
\lstinline'XAttributes' class represents the attributes of an element
as a map.)
\begin{lstlisting}
String getName() //element name
String setName(String name) 
XAttributes getAttributes()
List<XElement> getChildrenElements()
List<XElement> getChildrenElements(String elemName)
\end{lstlisting}

We can see the methods that allow us to get at the children elements
in action by examining \lstinline{evaluate()} in the
\lstinline'Add' class:
\begin{lstlisting} 
public class Add extends XElement
    implements Expression {
  public int evaluate(){
    int result=0;
    for(XElement elem: super.getChildrenElements()){
      if (elem instanceof Expression){
        result += ((Expression) elem).evaluate();
      }
    }
    return result;
  }
}
\end{lstlisting} 

This method iterates through each child element and casts each
reference to an \lstinline'Expression', an interface that requires the
\lstinline'evaluate()' method and is implemented by all calculator
classes.  Here we know ahead of time that we have mapped all elements
in the XML document to classes implementing this interface, so the
type cast will not fail.  Nevertheless, this approach can lead to
problems if we do not know all of the elements in the XML document
when we perform the compile-time mapping.  If this is the case, there
must be explicit checks or appropriate exception handling if the
parsing is to run to completion.

The actual parsing of the XML document and retrieval of the object
tree is facilitated by a utility class:
\begin{lstlisting}
Reader reader = new FileReader("expressions.xml");
XDocument document = 
    ParserUtil.getXElementFromXml(reader, handler);
Expression expr = 
    (Expression) document.getRootElement();  
\end{lstlisting}
This \lstinline{getXElementFromXml} method accepts the previously
mentioned event handler (which contains mapping information), and a
Java \lstinline{Reader} (implementations of which can read from files,
strings, etc). The returned \lstinline{XDocument} reference wraps the
object tree, and the root can be requested and type-cast if necessary.

\subsection{NaturalXML}
Unlike DOM or XElement, NaturalXML does not require objects to extend
a common base class.  Instead, objects are written as usual, with
their own class hierarchy if necessary, and metadata embedded within
the class describes the relationship to the markup.  In the reference
Java implementation, this metadata is implemented with field and class 
annotations.

Returning to the calculator example, the \lstinline'Add' class in
NaturalXML is expressed as follows:
\begin{lstlisting}
@Element("add")
public class Add extends ContainsExpressionList {
  public int evaluate() {
    int sum = 0;
    for(Expression expr: super.expressions) {
      sum += expr.evaluate();
    }
    return sum;
  }
}
\end{lstlisting}
Recalling that there is a one-to-one relationship between classes and
elements, we see that that the \lstinline{@Element} annotation on the
class itself specifies this mapping.  Thus when a document is parsed
the \lstinline'Add' class will be instantiated whenever an
\lstinline'add' element is encountered.  The logic in the
\lstinline'evaluate' method is the same as in the XElement version,
but there are no type casts.  This is the because the
\lstinline'expressions' instance variable is declared in the
superclass with extra type information (via generics) that makes the
casts unnecessary.  This super class is not part of the framework.
Rather, it contains the code common to all operations in the
calculator example.  Looking at the abstract super class:

\begin{lstlisting}
public abstract class ContainsExpressionList 
     implements Expression {
	
  @Children({Group.class, Add.class, 
             Subtract.class, Value.class})
  protected List<Expression> expressions = 
      new ArrayList<Expression>();
  
  public List<Expression> getExpressions() {
    return expressions;
  }
  
  public void addExpression(Expression expr) {
    expressions.add(expr);
  }
}   
\end{lstlisting}

We can see this extra type information and the annotation on the
\lstinline{expressions} field.  Elements mapped to the classes listed
in the \lstinline'@Children' annotation are all stored in the list,
and the type of the list specifies that each of these classes must
also be an \lstinline'Expression'.  In the current version of
NaturalXML this check is delayed until run time, but annotations can
be examined at compile time, and this is one possible area of future
improvement. Still, this approach is not only safer but also more
concise.

\subsubsection{Annotations}
Six annotations form the basis of the NaturalXML system, providing the
mechanism for binding elements to classes and fields to attributes, 
child elements, and character data.  Each
annotation is discussed below in detail:

\vspace*{0.5em}
\noindent\lstinline{@Element("element_name")}\\
This is the only class-level annotation in the framework, and it
defines the one-to-one mapping between classes and element names.  This
element name is the annotation's only parameter and is required.

\vspace*{0.5em}
\noindent\lstinline!@Children({SubElement1.class, SubElement2.class})!\\ 
This field-level annotation describes an instance variable that is
usually a collection of references to child element data
(i.e. instances of the classes in the annotation's parameter list).
Each class in the annotation's parameter list must contain an
\lstinline'@Element' annotation and is thus  mapped to some markup
element.  
The object hierarchy mimics the element hierarchy; when the elements
mapped to the classes in the parameter list are encountered (as 
sub-elements of the element to which the enclosing class is mapped) 
they are instantiated and populated.  The mutator for the field
is then called with this inflated sub-element object as a parameter.
The type of the field being
described is not important (but usually a subclass of
\lstinline{java.util.Collection}). 
NaturalXML uses the name of the
field to locate the appropriate accessor and mutator methods via
reflection.  The name of the accessor follows the Java Beans
specification~\cite{javaBeans}.  For example, with a field named
\lstinline'children', NaturalXML looks for an accessor named
\lstinline'getChildren()'.  The name of the mutator is the singular
form of the field name prepended with ``add''; with this same field,
NaturalXML looks for the mutator \lstinline'addChild(...)'.  A port of
the Ruby on Rails Inflector~\cite{inflector} provides the
singularization capability.

\vspace*{0.5em}
\noindent\lstinline!@Singleton!\\
This field-level annotation may occur only on elements already
annotated with \lstinline!@Children!.  It signifies that one and only
one child is expected and should annotate a reference to that child as
opposed to a collection (as in a standard \lstinline!@Children!
annotation).  Both the accessor and mutator (i.e. ``get'' and ``set'' methods)
are then discovered as per the Java Beans specification
~\cite{javaBeans}.

\vspace*{0.5em}
\noindent\lstinline!@Attribute("attribute_name")!\\
This field-level annotation describes instance variables of type
\lstinline'String'.  This field contains the value of the attribute specified in the parameter. Both the accessor and
mutator are then discovered as per the Java Beans
specification~\cite{javaBeans}.

\vspace*{0.5em}
\noindent\lstinline{@CData}\\
This field-level annotation describes a instance variables of type
\lstinline'List<CDataWrap>' and contains the character data of the
element mapped to the enclosing class.  Elements with character data
may provide only one field with this annotation.  The
\lstinline'CDataWrap' class is a string wrapper provided for
convenience with methods such as reducing a list
of \lstinline'CDataWrap' instances to one string.  It also provides a hook for
future expansion.

\vspace*{0.5em}
\noindent\lstinline{@Namespace("http://namespace.uri")}\\
This annotation is applied at the class or field level to specify
a namespace for elements or attributes.  It cannot stand alone.

\subsubsection{Usage}
The NaturalXML object tree is ideally suited to situations where the
underlying data model \emph{or} the underlying markup may change
frequently.  This may occur, for example, in exploratory programming,
prototyping, or the creation of ad-hoc data formats.  Similarly,
NaturalXML is useful when an existing data model needs to be expressed
in markup.  As shown in the calculator example, the changes are
simple, requiring little more than the insertion of the appropriate
annotations into an existing class hierarchy.

  Notice, however, that some of the burden does fall on the programmer
in NaturalXML; every piece of the markup must be represented in some
data structure lest it not be preserved.  In situations where the
programmer is only in interested in manipulating a small subset of
the data stored in the markup, this can be tedious and XElement would
be the preferable tree format.

\subsubsection{Limitations}
A limitation of NaturalXML in its current form is its inability to
retain ordering information among interspersed elements that are
mapped to different data structures.  As an example, consider the XML
fragment:
\begin{lstlisting} 
<root>
  <a num="1"/>
  <b num="2"/>
  <a num="3"/>
</root>
\end{lstlisting}

Assume we map the \lstinline{root} element to the \lstinline{Root}
class, the \lstinline{b} element to the \lstinline{B} class, and the
\lstinline{a} element to the \lstinline{A} class.  If in the
\lstinline{Root} class we store instances of \lstinline{A} and
\lstinline{B} in distinct lists:

\begin{lstlisting}
@Element("root")
public class Root {
  @Children(A.class)
  List<A> aList;

  @Children(B.class)
  List<B> bList;
}
\end{lstlisting}
Then the ordering of the \lstinline{a} elements relative to the
\lstinline{b} elements will not be preserved. That is, all the
\lstinline{a} elements will be printed before the \lstinline{b}
elements. If we were to output the NaturalXML tree to XML we would get:
\begin{lstlisting} 
<root>
  <a num="1"/>
  <a num="3"/>
  <b num="2"/>
</root>
\end{lstlisting}

While all element, attribute, and character data information is preserved, the
ordering information of elements mapped to different data structures
is not.  While this limitation is usually not an issue for the ad-hoc
data formats to which NaturalXML is ideally suited, %more elaboration
                                %here???
the limitation does make it difficult to represent documents like HTML
where such ordering information is often necessary and one would like
to store instances of the objects mapped to these elements in
disparate data structures.  If this is a requirement, it may be best
to use XElement, but we plan to alleviate this problem in the near
future by using an order-preserving map implementation for the
children.


\section{Event Streams}\label{streams}
Both of the object trees discussed in Section~\ref{sec:ootrees} are
unaware of the underlying markup.  Rather, NaturalXML and XElement
trees are created from a stream of events.  Likewise, each is
serialized to the same stream of events.  This approach decouples the
tree formats from the various input and output formats, much like
intermediate representations decouple the front end of a compiler from
the back end. As in the compiler world where $n$ front ends and $m$
back ends lead to $n*m$ possible configurations, the same holds in the
BetterXML framework for tree and markup formats.

%include comparison and contrast with SAX here

Internally, the event stream is a sequence of method calls and can be
externalized using the XML Intermediate Representation described in
Section~\ref{sec:xir}.  On the input side, each tree format provides
an implementation of the the \lstinline{BetterXmlContentHandler}
interface (all methods are \lstinline{public}):

\begin{lstlisting}
void startDocument()
void endDocument()
void startPrefixMapping(String prefix, String uri)
void endPrefixMapping(String prefix)
void startElement(String uri, String name, 
        String qname, int attrCount)
void endElement(String uri, String name, 
        String qname)
void attribute(String uri, String name, 
        String qname, String value)
void characters(int length, String cdata)
void whitespace(int length, String cdata)
void skippedEntity(String name)
void processingInstruction(String name, 
        String target)
\end{lstlisting}

For example, the \lstinline'NaturalXmlContentHandler' is an
implementation of this interface and creates a NaturalXML object tree.
When the \lstinline'startElement()' method is called, an instance of
the appropriate class (i.e. the class mapped to that element name) is
created and pushed onto a stack.  Future method calls adding children,
attributes, or character data are then able to find the parent class
by looking at the top of the stack.  After processing, the handler
provides a method to retrieve the root of the NaturalXML object tree,
which sits at the bottom of the stack.

Both tree formats are serialized to this same event stream.  In
XElement, the logic is embedded within the base class via the
\lstinline{acceptContentHandler()} method.  This implementation of the
visitor pattern~\cite{gangOf4} lets the tree accept and
serialize itself to any implementation of the aforementioned
\lstinline{BetterXmlContentHandler} interface.  NaturalXML trees are
serialized by traversing the tree externally with simple recursive method
that takes the tree and a handler as a parameter.  This approach is necessary as
users of NaturalXML should not be expected to embed traversal logic in
their objects.

In practice, the \lstinline'ToXMLContentHandler', which generates an
XML document from an event stream, is the most common event handler
for serialization.  However, we can hand any handler to the traversal
logic, and the flexibility allows us to generate other markup such as
XIR or do more unusual things such as generating another object tree
or a SAX event stream.




\section{XIR}\label{sec:xir}
%%%%%%%%%%%%%%%%%%%%%%%%%%%%%%%%%%%%%%%%
% XIR                                  %
%%%%%%%%%%%%%%%%%%%%%%%%%%%%%%%%%%%%%%%%

The XML Intermediate Representation (XIR) provides a structured
intermediate form of an XML document's content for efficient streaming
and other types of data exchange. XIR also serves as an externalized
representation of a document's BetterXML event stream discussed in
Section~\ref{streams}.

While XIR aims to be somewhat human readable, the important
distinction is that the format is not tree-structured but rather a
record-oriented regular language. Therefore, even though the resulting
linear representation appears less concise than the corresponding XML
tree, XIR is fast to process and interpret (similar to the byte-code
concept found in Java and elsewhere). The simplicity of the format
also makes the implementation of parsers trivial; it requires nothing
more than the ability to read colon-delimited text and the encoding
and decoding of Base64 data. Thus the format can be easily ported to
any language or platform.

As an example, consider the XML fragment below:
\begin{lstlisting}
<value value="5">some cdata</value>
\end{lstlisting}
This fragment is encoded in XIR as:
\begin{lstlisting}
xir.type:verbatim=element
ns:verbatim=
xir.subtype:verbatim=begin
qname:verbatim=value
name:verbatim=value
attributes:verbatim=1

xir.type:verbatim=attribute
ns:verbatim=
xir.subtype:verbatim=none
qname:verbatim=value
name:verbatim=value
value:verbatim=5

xir.type:verbatim=characters
cdata:base64=c29tZSBjZGF0YQ==
xir.subtype:verbatim=none
length:verbatim=10

xir.type:verbatim=element
ns:verbatim=
xir.subtype:verbatim=end
qname:verbatim=value
name:verbatim=value
\end{lstlisting}
The start of the element, its attribute, the character data, and the
end of the element are represented as distinct records.  Some records
(i.e. those representing elements or prefix mappings) have a beginning
and end so as to nest other records inside their scope. Others (i.e
attributes, whitespace and cdata) are ``singletons'' with no nested
scope. This behavior is specified in the \lstinline{xir.subtype}
field.  

Notice also that each field is contains a ``verbatim'' or ``base64''
identifier which specifies whether or not the value of the field is
encoded as plain text or in Base64.  In general, only character data
is encoded in Base64; this allows for compactness and maintains the
record-oriented format and ease of parsing by suppressing (but
losslessly retaining) newlines and whitespace. Ordering of fields
within a record is not specified nor should it be relied upon as the
fields are usually represented in memory by an unordered data
structure such as a hash table.

%\subsection{Performance}
%XIR Performance numbers coming soon...

\section{Related Work}
There is a wide range of related work, of which we attempt to describe
the most relevant and closely related.

Our work most closely overlaps with work on \emph{data binding}. In
particular, we focus on binding XML data to application-specific data
structures. In object-oriented programming, such data structures are
built using (hierarchically arranged) classes. There are a number of
efforts to address data binding as defined in this way, and we briefly
summarize a few of these.

One of the popular approaches is the JDOM (Java Document Object
Model)~\cite{jdom}, which is effectively an implementation of the
standard World Wide Web (W3) Consortium DOM implementation. While
written in Java, the JDOM does not support direct binding of elements
and attributes to classes and instance variables as done in the
BetterXML framework. XElement supports the \emph{essence} of W3's DOM
(while choosing not to be fully compatible) and in its default
implementation works exactly like JDOM.  Just as JDOM builds an object
tree of JDOM Element instances, so, too, will XElement build a tree of
\lstinline{XElement} instances, should the programmer choose not to
custom-map element names to classes.


JAXB~\cite{ort:JAXB03} is a Java framework that goes from XML schemas
to class hierarchies. BetterXML does not start at the level of
schemas, and we presently do not support code generation from a
schema. We take the view that code generation systems are only
effective if the underlying framework allows the generated code to be
modified and can pick up any subsequent changes that are made to the
schema. The JAXB team must be well-aware of the problem as their
latest effort includes a mechanism based on reflection and
annotations---ideas that have been in our implementation since
\cite{gkt:natXml04}. Nevertheless, we share a number of goals with the
JAXB effort and will be using the framework itself to develop a code
generator that goes from RELAX NG~\cite{relaxng} schemas to
NaturalXML.

JAXB's reflection API~\cite{jaxbReflect} is actually borrowed (almost
wholesale) from the Microsoft .NET framework~\cite{microsoftxml},
which makes direct use of annotations \lstinline{XmlAttribute} and
\lstinline{XmlElement} to identify elements and attributes. As shown
above, NaturalXML provides full support for ``all things XML'' and
allows programmers to map namespaces and address how character data
should be mapped to and from XML. NaturalXML also works clearly with
classes related by inheritance. Despite its simplicity, the expression
language from our example in Section~\ref{sec:motivation} is typical
for the widespread use of recursion in real XML languages and the need
to map to object oriented class hierarchies. Because most data-binding
frameworks fail to deliver support for this most fundamental and
common usage, many XML programmers resort to writing their own parser
\emph{handlers} using SAX to avoid the headaches. We believe our
framework is among the first to support recursive mapping properly.

Our work on an XML Intermediate Representation (XIR) is partially
covered by the SGML/ESIS (Element Structure, Information Set) work of
James Clark~\cite{sgml-esis}, which aims to have a non-SGML
formulation of any SGML document.  SGML is a predecessor to XML that
fundamentally differed in its lack of support for namespaces and its
more ambiguous structure. It was also hard to parse correctly because
closing an element was optional (much the same as in HTML version 4
and earlier). The Pyxie framework~\cite{pyxie} was an attempt to
resurrect many of ESIS ideas in XML but does not appear to be an
actively maintained effort. In addition, XIR goes well beyond the
ideas by supporting all XML features and ensuring that they are
encoded losslessly. Our ability to encode record fields in verbatim or
Base64 mode is a competitive advantage and allows us to guarantee that
character data and international text are encoded in a truly printable
and portable format. All told, we consider ESIS and Pyxie as important
inspiration; our work is a refinement that we believe makes it
possible to embed XML in any programming environment, especially when
the full power of an XML parser is neither desired nor needed.

% KL: "replayed multiple times" sounds like a pull model
% so how closely related is it to StAX?!? 
% http://en.wikipedia.org/wiki/StAX

It must be said that SAX (Simple API for XML Processing) is a
significant inspiration for our work. SAX by its nature focuses on the
essence of XML parsing and fulfills many of the same goals as a front
end does for a language compiler. SAX influenced our ideas on XIR,
which can be thought of as a record-oriented format that supports the
SAX event stream. The only difference is that (unlike SAX) XIR allows
the events to be replayed multiple times if necessary. Furthermore,
because XIR can itself be generated, we envision a day when many
applications could be written with XML \emph{in mind} but not
necessarily \emph{in sight}.  While our benchmarking efforts and
optimization are still under way, XIR requires fewer resources to
process than most XML parsers, and the stateless design makes it an
ideal match for distributed systems, which are increasingly relying
upon XML/RPC schemes. While beyond the scope of the present work, XIR
can stand alone as a separate serialization/deserialization system
to build a robust and efficient RPC system with all of the benefits of
full \mbox{XML/RPC} minus the significant overhead of building temporary data
structures.

Another library for serialization of Java objects to XML and back
designed with simplicity and ease of use in mind is
XStream~\cite{xstream}. XStream uses reflection to map Java classes to
XML elements automatically; in addition, it supports aliases and other
custom mappings, as well as annotations. Nevertheless, the crucial
difference between serialization libraries such XStream and more
general data binding frameworks, including BetterXML, is the ability
of the latter to start with existing XML documents as opposed to a
Java class hierarchy.

We have presented an earlier version of this work in a non
peer-reviewed department column on scientific
programming~\cite{gkt:natXml04}, where we presented an overview of the
use of reflection and POJOs (plain old Java objects). Our work here is
a synthesis of many frameworks and is part of an evolving vision to
build one of the most intuitive and efficient XML processing systems.

Finally, we know that the world is beginning to question how and when
XML should be used. YAML~\cite{yaml} is an approach that builds on the
notion of \emph{property files} and is as simple to parse as XIR. We
believe this model is mostly equivalent to XML but lacks support for
\emph{schemas}, which we believe will ultimately limit its success in
serious application development.  JavaScript Object Notation
(JSON)~\cite{json} is also gaining popularity in the new world of web
development (affectionately known as Web 2.0). It is used primarily in
handling \lstinline{XmlHttpRequest()} calls but is used widely for
initializing state for plugins and other JavaScript data
structures. While we believe JSON to be a good approach for this
purpose, the inability to do meaningful type checking and schema
validation will ultimately limit its use outside of web
applications. We mention these alternatives to make it clear that
there are alternatives to XML that, while tempting, are not as
robust. We are hoping our work will allow developers to reconsider
XML. When done right, XML can be lightweight and straightforward as
these competing approaches.


\section{Analysis and Conclusion}
The BetterXML framework allows programmers to focus on their domain
objects, and the decoupling of input and output formats from the
object tree interface via an internal event stream provides room for
future growth if necessary.  Though not suited for all uses,
NaturalXML is ideal for ad-hoc data formats, and XElement performs
well whenever a DOM-like tree is desired.  In addition, XIR provides a
record-oriented representation of XML documents for efficient
streaming and other forms of data exchange. We hope this work
represents a useful step forward in the processing of markup.

The Java reference implementation of BetterXML, including the examples
shown in this paper, is available as an open-source project through
Google Code at http://code.google.com/p/betterxml

% references section
\bibliographystyle{IEEEtran}
\bibliography{IEEEabrv,sources}

\end{document}


