%%%%%%%%%%%%%%%%%%%%%%%%%%%%%%%%%%%%%%%%
% XIR                                  %
%%%%%%%%%%%%%%%%%%%%%%%%%%%%%%%%%%%%%%%%

The XML Intermediate Representation (XIR) provides a structured
intermediate form of an XML document's content for efficient streaming
and other types of data exchange. XIR also serves as an externalized
representation of a document's BetterXML event stream discussed in
Section~\ref{streams}.

While XIR aims to be somewhat human readable, the important
distinction is that the format is not tree-structured but rather a
record-oriented regular language. Therefore, even though the resulting
linear representation appears less concise than the corresponding XML
tree, XIR is fast to process and interpret (similar to the byte-code
concept found in Java and elsewhere). The simplicity of the format
also makes the implementation of parsers trivial; it requires nothing
more than the ability to read colon-delimited text and the encoding
and decoding of Base64 data. Thus the format can be easily ported to
any language or platform.

As an example, consider the XML fragment below:
\begin{lstlisting}
<value value="5">some cdata</value>
\end{lstlisting}
This fragment is encoded in XIR as:
\begin{lstlisting}
xir.type:verbatim=element
ns:verbatim=
xir.subtype:verbatim=begin
qname:verbatim=value
name:verbatim=value
attributes:verbatim=1

xir.type:verbatim=attribute
ns:verbatim=
xir.subtype:verbatim=none
qname:verbatim=value
name:verbatim=value
value:verbatim=5

xir.type:verbatim=characters
cdata:base64=c29tZSBjZGF0YQ==
xir.subtype:verbatim=none
length:verbatim=10

xir.type:verbatim=element
ns:verbatim=
xir.subtype:verbatim=end
qname:verbatim=value
name:verbatim=value
\end{lstlisting}
The start of the element, its attribute, the character data, and the
end of the element are represented as distinct records.  Some records
(i.e. those representing elements or prefix mappings) have a beginning
and end so as to nest other records inside their scope. Others (i.e
attributes, whitespace and cdata) are ``singletons'' with no nested
scope. This behavior is specified in the \lstinline{xir.subtype}
field.  

Notice also that each field is contains a ``verbatim'' or ``base64''
identifier which specifies whether or not the value of the field is
encoded as plain text or in Base64.  In general, only character data
is encoded in Base64; this allows for compactness and maintains the
record-oriented format and ease of parsing by suppressing (but
losslessly retaining) newlines and whitespace. Ordering of fields
within a record is not specified nor should it be relied upon as the
fields are usually represented in memory by an unordered data
structure such as a hash table.

%\subsection{Performance}
%XIR Performance numbers coming soon...