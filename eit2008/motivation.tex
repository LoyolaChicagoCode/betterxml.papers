When processing markup in an object-oriented language, we often
confront a large amount of accidental complexity.  Instead of dealing
with our domain objects directly, we are forced to consider event
streams or DOM trees.  The BetterXML framework aims to mitigate the
situation, allowing for the processing of markup without losing focus
on the underlying domain model. The design goals of BetterXML are ease
of use, simplicity, flexibility, and grounding in design patterns.

The BetterXML framework starts from the assumption that markup is
inherently tree-structured.  Though this starting point may differ
from the original intent of markup as a stream of characters
interrupted by the occasional tag, the XML specification does not
allow for documents without tags and requires exactly one root
element~\cite{xmlSpec}.  Thus the specification itself is a
recognition of this tree structure. Following this assumption,
BetterXML represents markup in the form of an object tree, and the
framework contributes two types of these trees, XElement and
NaturalXML, as well as an internal event stream representation and a
record-oriented external representation.

\paragraph*{XElement} While both types of object trees retain a
one-to-one relationship between elements and classes, XElement
is the more DOM-like of the two.  In XElement, all classes extend a
common base class.  This base class contains a single heterogeneous
list of all children and a map of attributes.

\paragraph*{NaturalXML} Classes in NaturalXML, on the other
hand, need not extend a common base class.  Instead the mapping
between elements, attributes, and child elements occurs in class
metadata, and each class is responsible for providing its own data
structures for children and attributes.  

\paragraph*{Event streams} Both tree formats in the framework know
nothing of the underlying markup they represent; instead, the trees
are inflated from and serialized to an event stream.  This stream,
though not meant for end users, is the lingua franca of the system,
serving as an intermediary that decouples the object trees from the
various kinds of markup the framework supports.

\paragraph*{XIR} Though the framework was originally intended for XML
processing, it is compatible with other forms of markup including its
own representation, the XML Intermediate Representation (XIR), a
record-oriented regular language that is easy to parse and for which
writing a parser is trivial. XIR also serves as an external
representation of event streams.

In this paper we will discuss BetterXML as it relates to the reference
Java implementation. The ideas of BetterXML, while closely tied to the
object-oriented paradigm, are independent of specific object-oriented
languages, and Python and C\#/.NET ports of the framework are in
progress.

\subsection{A Brief Example}
Throughout this paper we will use a simple applicative calculator to
illustrate various aspects of the BetterXML framework.  The calculator
will support integer addition and subtraction, and we will represent
these operations in XML.  For example, we express the computation
$21+20+(5-4)$ as:

\begin{lstlisting}
<group>
  <add>
    <value value="21"/>
    <value value="20"/>
    <subtract>
      <value value="5"/>
      <value value="4"/>
    </subtract>
  </add>
</group>
\end{lstlisting}
The computation, of course, evaluates to $42$.  If we were to represent
this as an expression tree in an object-oriented language, we would
most likely create a distinct class for each operation.  Each
of these classes might share a common method, \lstinline{evaluate()}.
Values would evaluate to themselves, and a more complex operation like
addition would call the \lstinline{evaluate()} method on all children
and return their sum.

This example is typical of an introductory object-oriented programming
or data structures course and is used to illuminate the ideas of
polymorphism and recursive evaluation.  In such a setting, the example
is valued for its conceptual clarity.  In the sections below we will
show how the BetterXML framework can represent this problem in XML
while using the aforementioned object model with little or no
modification.
