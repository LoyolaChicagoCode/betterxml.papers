Both of the object trees discussed in Section~\ref{sec:ootrees} are
unaware of the underlying markup.  Rather, NaturalXML and XElement
trees are created from a stream of events.  Likewise, each is
serialized to the same stream of events.  This approach decouples the
tree formats from the various input and output formats, much like
intermediate representations decouple the front end of a compiler from
the back end. As in the compiler world where $n$ front ends and $m$
back ends lead to $n*m$ possible configurations, the same holds in the
BetterXML framework for tree and markup formats.

%include comparison and contrast with SAX here

Internally, the event stream is a sequence of method calls and can be
externalized using the XML Intermediate Representation described in
Section~\ref{sec:xir}.  On the input side, each tree format provides
an implementation of the the \lstinline{BetterXmlContentHandler}
interface (all methods are \lstinline{public}):

\begin{lstlisting}
void startDocument()
void endDocument()
void startPrefixMapping(String prefix, String uri)
void endPrefixMapping(String prefix)
void startElement(String uri, String name, 
        String qname, int attrCount)
void endElement(String uri, String name, 
        String qname)
void attribute(String uri, String name, 
        String qname, String value)
void characters(int length, String cdata)
void whitespace(int length, String cdata)
void skippedEntity(String name)
void processingInstruction(String name, 
        String target)
\end{lstlisting}

For example, the \lstinline'NaturalXmlContentHandler' is an
implementation of this interface and creates a NaturalXML object tree.
When the \lstinline'startElement()' method is called, an instance of
the appropriate class (i.e. the class mapped to that element name) is
created and pushed onto a stack.  Future method calls adding children,
attributes, or character data are then able to find the parent class
by looking at the top of the stack.  After processing, the handler
provides a method to retrieve the root of the NaturalXML object tree,
which sits at the bottom of the stack.

Both tree formats are serialized to this same event stream.  In
XElement, the logic is embedded within the base class via the
\lstinline{acceptContentHandler()} method.  This implementation of the
visitor pattern~\cite{gangOf4} lets the tree accept and
serialize itself to any implementation of the aforementioned
\lstinline{BetterXmlContentHandler} interface.  NaturalXML trees are
serialized by traversing the tree externally with simple recursive method
that takes the tree and a handler as a parameter.  This approach is necessary as
users of NaturalXML should not be expected to embed traversal logic in
their objects.

In practice, the \lstinline'ToXMLContentHandler', which generates an
XML document from an event stream, is the most common event handler
for serialization.  However, we can hand any handler to the traversal
logic, and the flexibility allows us to generate other markup such as
XIR or do more unusual things such as generating another object tree
or a SAX event stream.


