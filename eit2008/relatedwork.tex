There is a wide range of related work, of which we attempt to describe
the most relevant and closely related.

Our work most closely overlaps with work on \emph{data binding}. In
particular, we focus on binding XML data to application-specific data
structures. In object-oriented programming, such data structures are
built using (hierarchically arranged) classes. There are a number of
efforts to address data binding as defined in this way, and we briefly
summarize a few of these.

One of the popular approaches is the JDOM (Java Document Object
Model)~\cite{jdom}, which is effectively an implementation of the
standard World Wide Web (W3) Consortium DOM implementation. While
written in Java, the JDOM does not support direct binding of elements
and attributes to classes and instance variables as done in the
BetterXML framework. XElement supports the \emph{essence} of W3's DOM
(while choosing not to be fully compatible) and in its default
implementation works exactly like JDOM.  Just as JDOM builds an object
tree of JDOM Element instances, so, too, will XElement build a tree of
\lstinline{XElement} instances, should the programmer choose not to
custom-map element names to classes.


JAXB~\cite{ort:JAXB03} is a Java framework that goes from XML schemas
to class hierarchies. BetterXML does not start at the level of
schemas, and we presently do not support code generation from a
schema. We take the view that code generation systems are only
effective if the underlying framework allows the generated code to be
modified and can pick up any subsequent changes that are made to the
schema. The JAXB team must be well-aware of the problem as their
latest effort includes a mechanism based on reflection and
annotations---ideas that have been in our implementation since
\cite{gkt:natXml04}. Nevertheless, we share a number of goals with the
JAXB effort and will be using the framework itself to develop a code
generator that goes from RELAX NG~\cite{relaxng} schemas to
NaturalXML.

JAXB's reflection API~\cite{jaxbReflect} is actually borrowed (almost
wholesale) from the Microsoft .NET framework~\cite{microsoftxml},
which makes direct use of annotations \lstinline{XmlAttribute} and
\lstinline{XmlElement} to identify elements and attributes. As shown
above, NaturalXML provides full support for ``all things XML'' and
allows programmers to map namespaces and address how character data
should be mapped to and from XML. NaturalXML also works clearly with
classes related by inheritance. Despite its simplicity, the expression
language from our example in Section~\ref{sec:motivation} is typical
for the widespread use of recursion in real XML languages and the need
to map to object oriented class hierarchies. Because most data-binding
frameworks fail to deliver support for this most fundamental and
common usage, many XML programmers resort to writing their own parser
\emph{handlers} using SAX to avoid the headaches. We believe our
framework is among the first to support recursive mapping properly.

Our work on an XML Intermediate Representation (XIR) is partially
covered by the SGML/ESIS (Element Structure, Information Set) work of
James Clark~\cite{sgml-esis}, which aims to have a non-SGML
formulation of any SGML document.  SGML is a predecessor to XML that
fundamentally differed in its lack of support for namespaces and its
more ambiguous structure. It was also hard to parse correctly because
closing an element was optional (much the same as in HTML version 4
and earlier). The Pyxie framework~\cite{pyxie} was an attempt to
resurrect many of ESIS ideas in XML but does not appear to be an
actively maintained effort. In addition, XIR goes well beyond the
ideas by supporting all XML features and ensuring that they are
encoded losslessly. Our ability to encode record fields in verbatim or
Base64 mode is a competitive advantage and allows us to guarantee that
character data and international text are encoded in a truly printable
and portable format. All told, we consider ESIS and Pyxie as important
inspiration; our work is a refinement that we believe makes it
possible to embed XML in any programming environment, especially when
the full power of an XML parser is neither desired nor needed.

% KL: "replayed multiple times" sounds like a pull model
% so how closely related is it to StAX?!? 
% http://en.wikipedia.org/wiki/StAX

It must be said that SAX (Simple API for XML Processing) is a
significant inspiration for our work. SAX by its nature focuses on the
essence of XML parsing and fulfills many of the same goals as a front
end does for a language compiler. SAX influenced our ideas on XIR,
which can be thought of as a record-oriented format that supports the
SAX event stream. The only difference is that (unlike SAX) XIR allows
the events to be replayed multiple times if necessary. Furthermore,
because XIR can itself be generated, we envision a day when many
applications could be written with XML \emph{in mind} but not
necessarily \emph{in sight}.  While our benchmarking efforts and
optimization are still under way, XIR requires fewer resources to
process than most XML parsers, and the stateless design makes it an
ideal match for distributed systems, which are increasingly relying
upon XML/RPC schemes. While beyond the scope of the present work, XIR
can stand alone as a separate serialization/deserialization system
to build a robust and efficient RPC system with all of the benefits of
full \mbox{XML/RPC} minus the significant overhead of building temporary data
structures.

Another library for serialization of Java objects to XML and back
designed with simplicity and ease of use in mind is
XStream~\cite{xstream}. XStream uses reflection to map Java classes to
XML elements automatically; in addition, it supports aliases and other
custom mappings, as well as annotations. Nevertheless, the crucial
difference between serialization libraries such XStream and more
general data binding frameworks, including BetterXML, is the ability
of the latter to start with existing XML documents as opposed to a
Java class hierarchy.

We have presented an earlier version of this work in a non
peer-reviewed department column on scientific
programming~\cite{gkt:natXml04}, where we presented an overview of the
use of reflection and POJOs (plain old Java objects). Our work here is
a synthesis of many frameworks and is part of an evolving vision to
build one of the most intuitive and efficient XML processing systems.

Finally, we know that the world is beginning to question how and when
XML should be used. YAML~\cite{yaml} is an approach that builds on the
notion of \emph{property files} and is as simple to parse as XIR. We
believe this model is mostly equivalent to XML but lacks support for
\emph{schemas}, which we believe will ultimately limit its success in
serious application development.  JavaScript Object Notation
(JSON)~\cite{json} is also gaining popularity in the new world of web
development (affectionately known as Web 2.0). It is used primarily in
handling \lstinline{XmlHttpRequest()} calls but is used widely for
initializing state for plugins and other JavaScript data
structures. While we believe JSON to be a good approach for this
purpose, the inability to do meaningful type checking and schema
validation will ultimately limit its use outside of web
applications. We mention these alternatives to make it clear that
there are alternatives to XML that, while tempting, are not as
robust. We are hoping our work will allow developers to reconsider
XML. When done right, XML can be lightweight and straightforward as
these competing approaches.
