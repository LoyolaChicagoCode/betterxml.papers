%%%%%%%%%%%%%%%%%%%%%%%%%%%%%%%%%%%%%%%%
% Object Trees                         %
%%%%%%%%%%%%%%%%%%%%%%%%%%%%%%%%%%%%%%%%

In this section, we discuss the two object tree formats provided by
BetterXML in more detail.

\subsection{XElement}
Of the two tree formats in BetterXML, XElement is the more DOM-like.
As in DOM, XElement classes must extend a common base class, and this
base class contains the data structures and methods for manipulating
children, attributes, and character data.  DOM, however, is not
designed for the programmer to create element-specific or other custom
subclasses. By contrast, XElement allows the programmer to map
specific XML elements to one particular subclass of the common base
class, thereby creating a domain-specific object tree.

For instance, in the calculator example we map the \lstinline'add',
\lstinline'subtract', \lstinline'group', and \lstinline'value'
elements to the \lstinline'Add', \lstinline'Subtract',
\lstinline'Group', and \lstinline'Value' classes respectively:
\begin{lstlisting}
ToXElementContentHandler handler = 
        new ToXElementContentHandler();  
handler.registerElementClass(Add.class, "add");
handler.registerElementClass(Subtract.class, 
                             "subtract");
handler.registerElementClass(Group.class, "group");
handler.registerElementClass(Value.class, value");  
\end{lstlisting}

Here, \lstinline'handler' is a reference to the event handler for the
BetterXML event stream (see section \ref{streams}), and this handler
is used to establish the mappings from element names to classes prior
to parsing a document. When the document is parsed and an element is
encountered, the mappings are examined.  If the element is mapped to a
particular class, then that class is instantiated.  Otherwise, the
base class, \lstinline'XElement', is instantiated.

Since all unmapped elements are represented as instances of 
\lstinline'XElement', the mappings from element names to classes
are optional.  Thus, smaller applications that lack a rich
domain model can rely directly on instances of \lstinline'XElement'.
The \lstinline'XElement' class itself is straightforward and has
methods for extracting the element name, attributes, and
children. Some methods of this class are excerpted below. (The
\lstinline'XAttributes' class represents the attributes of an element
as a map.)
\begin{lstlisting}
String getName() //element name
String setName(String name) 
XAttributes getAttributes()
List<XElement> getChildrenElements()
List<XElement> getChildrenElements(String elemName)
\end{lstlisting}

We can see the methods that allow us to get at the children elements
in action by examining \lstinline{evaluate()} in the
\lstinline'Add' class:
\begin{lstlisting} 
public class Add extends XElement
    implements Expression {
  public int evaluate(){
    int result=0;
    for(XElement elem: super.getChildrenElements()){
      if (elem instanceof Expression){
        result += ((Expression) elem).evaluate();
      }
    }
    return result;
  }
}
\end{lstlisting} 

This method iterates through each child element and casts each
reference to an \lstinline'Expression', an interface that requires the
\lstinline'evaluate()' method and is implemented by all calculator
classes.  Here we know ahead of time that we have mapped all elements
in the XML document to classes implementing this interface, so the
type cast will not fail.  Nevertheless, this approach can lead to
problems if we do not know all of the elements in the XML document
when we perform the compile-time mapping.  If this is the case, there
must be explicit checks or appropriate exception handling if the
parsing is to run to completion.

The actual parsing of the XML document and retrieval of the object
tree is facilitated by a utility class:
\begin{lstlisting}
Reader reader = new FileReader("expressions.xml");
XDocument document = 
    ParserUtil.getXElementFromXml(reader, handler);
Expression expr = 
    (Expression) document.getRootElement();  
\end{lstlisting}
This \lstinline{getXElementFromXml} method accepts the previously
mentioned event handler (which contains mapping information), and a
Java \lstinline{Reader} (implementations of which can read from files,
strings, etc). The returned \lstinline{XDocument} reference wraps the
object tree, and the root can be requested and type-cast if necessary.

\subsection{NaturalXML}
Unlike DOM or XElement, NaturalXML does not require objects to extend
a common base class.  Instead, objects are written as usual, with
their own class hierarchy if necessary, and metadata embedded within
the class describes the relationship to the markup.  In the reference
Java implementation, this metadata is implemented with field and class 
annotations.

Returning to the calculator example, the \lstinline'Add' class in
NaturalXML is expressed as follows:
\begin{lstlisting}
@Element("add")
public class Add extends ContainsExpressionList {
  public int evaluate() {
    int sum = 0;
    for(Expression expr: super.expressions) {
      sum += expr.evaluate();
    }
    return sum;
  }
}
\end{lstlisting}
Recalling that there is a one-to-one relationship between classes and
elements, we see that that the \lstinline{@Element} annotation on the
class itself specifies this mapping.  Thus when a document is parsed
the \lstinline'Add' class will be instantiated whenever an
\lstinline'add' element is encountered.  The logic in the
\lstinline'evaluate' method is the same as in the XElement version,
but there are no type casts.  This is the because the
\lstinline'expressions' instance variable is declared in the
superclass with extra type information (via generics) that makes the
casts unnecessary.  This super class is not part of the framework.
Rather, it contains the code common to all operations in the
calculator example.  Looking at the abstract super class:

\begin{lstlisting}
public abstract class ContainsExpressionList 
     implements Expression {
	
  @Children({Group.class, Add.class, 
             Subtract.class, Value.class})
  protected List<Expression> expressions = 
      new ArrayList<Expression>();
  
  public List<Expression> getExpressions() {
    return expressions;
  }
  
  public void addExpression(Expression expr) {
    expressions.add(expr);
  }
}   
\end{lstlisting}

We can see this extra type information and the annotation on the
\lstinline{expressions} field.  Elements mapped to the classes listed
in the \lstinline'@Children' annotation are all stored in the list,
and the type of the list specifies that each of these classes must
also be an \lstinline'Expression'.  In the current version of
NaturalXML this check is delayed until run time, but annotations can
be examined at compile time, and this is one possible area of future
improvement. Still, this approach is not only safer but also more
concise.

\subsubsection{Annotations}
Six annotations form the basis of the NaturalXML system, providing the
mechanism for binding elements to classes and fields to attributes, 
child elements, and character data.  Each
annotation is discussed below in detail:

\vspace*{0.5em}
\noindent\lstinline{@Element("element_name")}\\
This is the only class-level annotation in the framework, and it
defines the one-to-one mapping between classes and element names.  This
element name is the annotation's only parameter and is required.

\vspace*{0.5em}
\noindent\lstinline!@Children({SubElement1.class, SubElement2.class})!\\ 
This field-level annotation describes an instance variable that is
usually a collection of references to child element data
(i.e. instances of the classes in the annotation's parameter list).
Each class in the annotation's parameter list must contain an
\lstinline'@Element' annotation and is thus  mapped to some markup
element.  
The object hierarchy mimics the element hierarchy; when the elements
mapped to the classes in the parameter list are encountered (as 
sub-elements of the element to which the enclosing class is mapped) 
they are instantiated and populated.  The mutator for the field
is then called with this inflated sub-element object as a parameter.
The type of the field being
described is not important (but usually a subclass of
\lstinline{java.util.Collection}). 
NaturalXML uses the name of the
field to locate the appropriate accessor and mutator methods via
reflection.  The name of the accessor follows the Java Beans
specification~\cite{javaBeans}.  For example, with a field named
\lstinline'children', NaturalXML looks for an accessor named
\lstinline'getChildren()'.  The name of the mutator is the singular
form of the field name prepended with ``add''; with this same field,
NaturalXML looks for the mutator \lstinline'addChild(...)'.  A port of
the Ruby on Rails Inflector~\cite{inflector} provides the
singularization capability.

\vspace*{0.5em}
\noindent\lstinline!@Singleton!\\
This field-level annotation may occur only on elements already
annotated with \lstinline!@Children!.  It signifies that one and only
one child is expected and should annotate a reference to that child as
opposed to a collection (as in a standard \lstinline!@Children!
annotation).  Both the accessor and mutator (i.e. ``get'' and ``set'' methods)
are then discovered as per the Java Beans specification
~\cite{javaBeans}.

\vspace*{0.5em}
\noindent\lstinline!@Attribute("attribute_name")!\\
This field-level annotation describes instance variables of type
\lstinline'String'.  This field contains the value of the attribute specified in the parameter. Both the accessor and
mutator are then discovered as per the Java Beans
specification~\cite{javaBeans}.

\vspace*{0.5em}
\noindent\lstinline{@CData}\\
This field-level annotation describes a instance variables of type
\lstinline'List<CDataWrap>' and contains the character data of the
element mapped to the enclosing class.  Elements with character data
may provide only one field with this annotation.  The
\lstinline'CDataWrap' class is a string wrapper provided for
convenience with methods such as reducing a list
of \lstinline'CDataWrap' instances to one string.  It also provides a hook for
future expansion.

\vspace*{0.5em}
\noindent\lstinline{@Namespace("http://namespace.uri")}\\
This annotation is applied at the class or field level to specify
a namespace for elements or attributes.  It cannot stand alone.

\subsubsection{Usage}
The NaturalXML object tree is ideally suited to situations where the
underlying data model \emph{or} the underlying markup may change
frequently.  This may occur, for example, in exploratory programming,
prototyping, or the creation of ad-hoc data formats.  Similarly,
NaturalXML is useful when an existing data model needs to be expressed
in markup.  As shown in the calculator example, the changes are
simple, requiring little more than the insertion of the appropriate
annotations into an existing class hierarchy.

  Notice, however, that some of the burden does fall on the programmer
in NaturalXML; every piece of the markup must be represented in some
data structure lest it not be preserved.  In situations where the
programmer is only in interested in manipulating a small subset of
the data stored in the markup, this can be tedious and XElement would
be the preferable tree format.

\subsubsection{Limitations}
A limitation of NaturalXML in its current form is its inability to
retain ordering information among interspersed elements that are
mapped to different data structures.  As an example, consider the XML
fragment:
\begin{lstlisting} 
<root>
  <a num="1"/>
  <b num="2"/>
  <a num="3"/>
</root>
\end{lstlisting}

Assume we map the \lstinline{root} element to the \lstinline{Root}
class, the \lstinline{b} element to the \lstinline{B} class, and the
\lstinline{a} element to the \lstinline{A} class.  If in the
\lstinline{Root} class we store instances of \lstinline{A} and
\lstinline{B} in distinct lists:

\begin{lstlisting}
@Element("root")
public class Root {
  @Children(A.class)
  List<A> aList;

  @Children(B.class)
  List<B> bList;
}
\end{lstlisting}
Then the ordering of the \lstinline{a} elements relative to the
\lstinline{b} elements will not be preserved. That is, all the
\lstinline{a} elements will be printed before the \lstinline{b}
elements. If we were to output the NaturalXML tree to XML we would get:
\begin{lstlisting} 
<root>
  <a num="1"/>
  <a num="3"/>
  <b num="2"/>
</root>
\end{lstlisting}

While all element, attribute, and character data information is preserved, the
ordering information of elements mapped to different data structures
is not.  While this limitation is usually not an issue for the ad-hoc
data formats to which NaturalXML is ideally suited, %more elaboration
                                %here???
the limitation does make it difficult to represent documents like HTML
where such ordering information is often necessary and one would like
to store instances of the objects mapped to these elements in
disparate data structures.  If this is a requirement, it may be best
to use XElement, but we plan to alleviate this problem in the near
future by using an order-preserving map implementation for the
children.
