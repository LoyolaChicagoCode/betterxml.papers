\usepackage{fullpage}

\usepackage{listings}
\lstset{language=Java,
        basicstyle=\small}

\usepackage{graphicx}
\usepackage{hyperref}

\hypersetup{
  colorlinks=true,
  urlcolor=blue,
  linkcolor=black,
  pdftitle={NaturalXML},
  pdfauthor={Matt Bone and George Thiruvathukal}
}

\title{NaturalXML: Class Metadata for XML Serialization}
\author{Matt Bone and George Thiruvathukal}
\date{\today}

\begin{document}

\maketitle

\begin{abstract}
NaturalXML is a framework for manipulating XML in object oriented
programming languages.  Regular classes are simply marked up with
metadata that specifies how the class is to be serialized to, and
inflated from XML.  Objects are manipulated as usual, using the
language's collections classes.  The reference implemenetation
discussed in this paper is in Java, and the metadata is expressed via
class and field annotations.
\end{abstract}

\tableofcontents

\pagebreak

%\setlength{\parskip}{.25cm}
\bibliographystyle{abbrvnat}

\section{Motivation}
\subsection{XML for ad-hoc data formats}
XML originally as stream of characters interupted with markup, now as
a way of structuring stream of characters. %citation here?
However, many use XML not as markup, but rather as a data format.
While this frees the programmer from writing his or her own parser,
tradition XML tools are not adequate to meet this need.  DOM differes
greatly from the traditonal object model, and may be too heavy
weight. There is often need to implment an adapter between the DOM
objects and the natie data objects.  SAX, however, is often too
lightweight, and often forces the programmer to write his or her own
pushdown, the problem the programmer was initially trying to avoid.
NaturalXML offers an alternative


\section{Just Use Your Objects}
Unlike DOM or SAX, NaturalXML just lets you annotate and use your
existing data classes (or model, or business objects, etc).  You
simply write the classes as you would for a non-persistent
application, and then annotate your classes.  The annotations contain
the information about how the object will be represented in xml.
Besides these annotations, the data objects have no reliance on
outside code (i.e. they do not need to extend a particular class
or implement a particular interface).

The generated xml not for human consumption, and if you take input in, and
generate xml, you aren't guarenteed the same xml on the out.  Though the 
content will be preserved (i.e. you get the equivalent objects if you 
inflate from the same XML) the ordering of the elements is not.

As an example, consider the xml shown in listing \ref{alg:xml}.  With
the classeses and the annotations shown in listings \ref{alg:friends},
\ref{alg:friend}, and \ref{alg:foe} it is simple to create instances
of these classes from the xml and, in turn, generate the xml from these 
instances (show how to do this).

\begin{lstlisting}[float,caption={Friends, Friend and Foe in XML.},label=alg:xml,captionpos=b, frame=single, frameround=tttt]
<Friends>
   <Friend nick_name="george" number="773-555-1234">
    George K. Thiruvathukal
   </Friend>
   <Foe nick_name="matt" number="701-555-1111">
    Matt Bone
   </Foe>
</Friends>
\end{lstlisting}

\section{Implementation}
\noindent TODO:
\begin{enumerate}
 \item Reflection
 \item java beans specification discussion
\end{enumerate}

\section{Future Work}
\noindent TODO:
\begin{enumerate}
  \item Simple schema language, a domain specific language
  \item optional ordering of elements if objects implement an interface.
\end{enumerate}

\noindent This is a stub with a citation\cite{fowler03:peaa}

\begin{lstlisting}[float,caption={Friends class.},label=alg:friends,captionpos=b, frame=single, frameround=tttt]
  public class Friends {

    @Children(Friend.class)
    private ArrayList<Friend> friends = new ArrayList<Friend>();
    
    @Children(Foe.class)
    private ArrayList<Foe> foes = new ArrayList<Foe>();
    
    public List<Friend> getFriends(){ return Collections.unmodifiableList(friends); }
    public void addFriend(Friend friend) { friends.add(friend); }
    
    public List<Foe> getFoes(){ return Collections.unmodifiableList(foes); }
    public void addFoe(Foe foe) { foes.add(foe); }
}
\end{lstlisting}

\begin{lstlisting}[float,caption={Friend class.},label=alg:friend,captionpos=b, frame=single, frameround=tttt]
  @Element(name="Friend")
  public class Friend {

    @CData 
    private List<CDataWrap> name = new ArrayList<CDataWrap>();

    @Attribute("nick_name") 
    private String nickname;    

    @Attribute("number") 
    private String telephone;
    
    public List<CDataWrap> getName() { return Collections.unmodifiableList(name); }
    public void addName(String name) { this.name.add(new CDataWrap(name)); }

    public String getNickname() { return nickname; }
    public void setNickname(String nickname) { this.nickname = nickname; }

    public String getTelephone() { return telephone; }
    public void setTelephone(String telephone) { this.telephone = telephone; }
    
  }
\end{lstlisting}

\begin{lstlisting}[float,caption={Foe class.},label=alg:foe,captionpos=b, frame=single, frameround=tttt]
  @Element(name="Foe")
  public class Foe { 

    @CData 
    private List<CDataWrap> name = new ArrayList<CDataWrap>();

    @Attribute("nick_name") 
    private String nickname;

    @Attribute("number") 
    private String telephone;
    
    public List<CDataWrap> getName() { return Collections.unmodifiableList(name); }
    public void addName(String name) { this.name.add(new CDataWrap(name)); }

    public String getNickname() { return nickname; }
    public void setNickname(String nickname) { this.nickname = nickname; }

    public String getTelephone() { return telephone; }
    public void setTelephone(String telephone) { this.telephone = telephone; }
  }
\end{lstlisting}



%\begin{figure}
%  \center{\includegraphics[width=3in]{phlombay_class_diagram}}
%  \caption{\label{fig:phlombay-model}phlombay class diagram.}
%\end{figure}


\bibliography{sources}

\end{document}
